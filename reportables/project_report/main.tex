%-----------------------------------------------------------------------------
% Template for seminar 'Program Analysis' at TU Darmstadt.
%
% Adapted from template for sigplanconf LaTeX Class, which is a LaTeX 2e
% class file for SIGPLAN conference proceedings (by Paul C.
% Anagnostopoulos).
%
%-----------------------------------------------------------------------------


\documentclass[authoryear,preprint]{sigplanconf}

% A couple of packages that may be useful
\usepackage{amsmath}
\usepackage{amsfonts}
\usepackage{amssymb}
\usepackage{amsthm}
\usepackage{algorithm2e}
\usepackage{listings}
\usepackage{xcolor}
\usepackage{tikz}
\usepackage{booktabs}
\usepackage{subfigure}
\usepackage[english]{babel}
\usepackage{blindtext}
\usepackage[normalem]{ulem}

\begin{document}

\special{papersize=a4}
\setlength{\pdfpageheight}{\paperheight}
\setlength{\pdfpagewidth}{\paperwidth}


\title{System Dependence Graphs for Java Programs}

\authorinfo{Pavlos Milaszewicz, \sout{Ito Franchilo Mikael Alcuaz}}{}{}

\maketitle

\begin{abstract}
\blindtext % replace this with your own text
\end{abstract}


\section{Introduction}
\label{sec:introduction}

This is the introduction.


\section{Some Section}
\label{sec:some_section}

After this great introduction (Section~\ref{sec:introduction}), the 
following provides some additional hints and examples for the
layout and style of this paper. 


\subsection{Citations}

Use citations to refer to other 
papers~\cite{HerlihyMoss1993-TransactionalMemory,FraserHanson1992-CodeGenerator} 
and books~\cite{Strunk-ElementsOfStyle,Aho86-Compilers}.


\subsection{Tables}

Table~\ref{t:Translations} shows how a table looks like.

\begin{table}[ht]
\centering
\begin{tabular}{ll}
\hline
\textbf{English} & \textbf{German}\\
\hline
cell phone       & Handy\\
Diet Coke        & Coca Cola light\\
\hline
\end{tabular}
\caption[Translations]{\label{t:Translations}Translations.}
\end{table}

\subsection{Figures}

Figure~\ref{f:SOLAlogo} shows a simple figure with a single picture
and Figure~\ref{f:SubfigureExample} shows a more complex figure
containing subfigures.

\begin{figure}[ht]
\centering
\includegraphics[width=.6\linewidth]{figures/SOLALogo}
\caption[SOLA logo]{\label{f:SOLAlogo}SOLA logo.}
\end{figure}

\begin{figure}[ht]
\centering
\subfigure[TUDaLogo]{\includegraphics[height=12mm]{figures/TUDaLogo}}\quad
\subfigure[SOLALogo]{\includegraphics[height=12mm]{figures/SOLALogo}}
\caption[Subfigure example]{\label{f:SubfigureExample}Two pictures as
  part of a single figure through the magic of the subfigure package.}
\end{figure}


\subsection{Source code}

The listings package provides tools to typeset source code
listings. It supports many programming languages and provides a lot of
formatting options.

\lstset{numbers=left, numberstyle=\tiny, stepnumber=1, numbersep=5pt}
\lstset{basicstyle=\ttfamily}
\lstset{frame=tb}

\begin{lstlisting}[float,caption=Example usage of the listing package,label=l:javaClass,language=Java]
class S {
   int f1 = 42;
   public S(int x) {
          f1 = x;
   }
}
\end{lstlisting}

Listing \ref{l:javaClass} shows an example listing. Code snippets can
also be inserted in normal text:
\verb$\lstinline|int f1 = 42;|$ gives \lstinline$int f1 = 42;$


\subsection{Miscellany}

\begin{description}

\item[Capitalization.] When referring to a named table (such as in the
  previous section), the word \emph{table} is capitalized. The same is
  true for figures, chapters and sections.

\item[Bibliography.] Use \verb|bibtex| to make your life easier and to
  produce consistently formatted entries.

\item[Contractions.] Avoid contractions. For instance, use ``do not''
  rather than ``don't.''

\item[Style guide.] A classic reference book on writing style is
  Strunk's \emph{The Elements of Style} \cite{Strunk-ElementsOfStyle}.

\end{description}


\section{Another Section}

\blindtext % replace this with your own text


\section{Yet Another Section}

\blindtext % replace this with your own text


\section{Conclusion}

\blindtext % replace this with your own text

\bibliographystyle{abbrvnat}
\bibliography{references}


\bibliographystyle{abbrvnat}



\end{document}
